\documentclass[12pt,a4paper]{article}
\usepackage[utf8]{inputenc}
\usepackage[polish]{babel}
\usepackage{polski}
\usepackage{pgfplots}
\usepackage{listings}
\lstset{language=Java}
\usepackage{hyperref}
\usepackage{tikz}
\usetikzlibrary{arrows,shapes}
\pgfplotsset{width=10cm}
\usepackage[left=1cm,right=1cm,top=2cm,bottom=2cm]{geometry}
	\usepackage{fancyhdr}
	\pagestyle{fancy}
	\lhead{Inżynieria oprogramowania} % określa lewą część nagłówka
	\chead{PeerSharing} % określa środkową część nagłówka
	\rhead{Zdzisław Perkowski} % określa prawą część nagłówka
	\lfoot{} % określa lewą część stopki
	\cfoot{\thepage} % określa środkową część stopki
	\rfoot{} % określa prawą część stopki
\author{Zdzisław Perkowski}
\title{PeerSharing}

\begin{document}

\tikzstyle{vertex}=[circle,fill=black!25,minimum size=20pt,inner sep=0pt]
\tikzstyle{selected vertex} = [vertex, fill=red!24]
\tikzstyle{edge} = [draw,thick,-,black!50]
\tikzstyle{weight} = [font=\tiny, color=black]
\tikzstyle{selected edge} = [draw,line width=5pt,-,red!50]
\tikzstyle{ignored edge} = [draw,line width=5pt,-,black!30]

\maketitle
\tableofcontents
\pagebreak


\section{Opis programu}

\section{Wymagania funkcjonalne}

\section{Wymagania pozafunkcjonalne}

\section{Makiety}

\end{document}