\documentclass[12pt,a4paper]{article}
\usepackage[utf8]{inputenc}
\usepackage[polish]{babel}
\usepackage{polski}
\usepackage{pgfplots}
\usepackage{listings}
\lstset{language=Java}
\usepackage{hyperref}
\usepackage{tikz}
\usetikzlibrary{arrows,shapes}
\pgfplotsset{width=10cm}
\usepackage[left=1cm,right=1cm,top=2cm,bottom=2cm]{geometry}
	\usepackage{fancyhdr}
	\pagestyle{fancy}
	\lhead{Inżynieria oprogramowania} % określa lewą część nagłówka
	\chead{PeerSharing} % określa środkową część nagłówka
	\rhead{Zdzisław Perkowski} % określa prawą część nagłówka
	\lfoot{} % określa lewą część stopki
	\cfoot{\thepage} % określa środkową część stopki
	\rfoot{} % określa prawą część stopki
\author{Zdzisław Perkowski}
\title{PeerSharing}

\begin{document}

\tikzstyle{vertex}=[circle,fill=black!25,minimum size=20pt,inner sep=0pt]
\tikzstyle{selected vertex} = [vertex, fill=red!24]
\tikzstyle{edge} = [draw,thick,-,black!50]
\tikzstyle{weight} = [font=\tiny, color=black]
\tikzstyle{selected edge} = [draw,line width=5pt,-,red!50]
\tikzstyle{ignored edge} = [draw,line width=5pt,-,black!30]

\maketitle
\tableofcontents
\pagebreak


\section{Opis programu}

Program \textit{PeerSharing} pozwala na współdzielenie plików między różnymi urządzeniami przez sieć.

\begin{center}
\href{https://github.com/zperkowski/PeerSharing}{GitHub.com/zperkowski/PeerSharing}
\end{center}

\textit{PeerSharing} jest programem działającym na systemy operacyjne:
\begin{itemize}
	\item Windows 7 lub nowszy
	\item macOS w wersji 10.12 lub nowszej
	\item Android w wersji 1.6 lub nowszej
\end{itemize}

W dalszej części dokumentacji program \textit{PeerSharing} będzie nazywany \textit{programem}.

\subsection{Opis pojęć}

W dalszej części dokumentacji zostały przedstawione opisy wymagań w których są użyte określenia opisane poniżej:

\begin{description}
	\item[Komputer] to urządzenie na którym jest zainstalowany jeden z kompatybilnych systemów operacyjnych: Windows, Linux lub macOS. Urządzenie to również posiada zainstalowany program i jest podłączone do sieci LAN i ma dostęp do Internetu.
	\item[Smartphone] to urządzenie mobilne, które posiada zainstalowany jeden z mobilnych systemów operacyjnych, który jest kompatybilny z programem: Android lub iOS. Jednocześnie na tym urządzeniu jest zainstalowany program oraz urządzenie jest podłączone do sieci LAN i ma dostęp do Internetu.
	\item[Urządzenie] to komputer lub smartphone.
	\item[Użytkownik] jest to osoba aktualnie obsługująca program za pomocą komputera lub smartphona.
	\item[Program] to program \textit{PeerSharing}, która aktualnie działa na urządzeniu użytkownika i może wykonywać operacje w tle.

\end{description}

\section{Wymagania funkcjonalne}
W tej sekcji znajduje się opis wymagań funkcjonalnych. Wiersz oznaczony gwiazdką (\textbf{*}) oznacza, że jest to czynność opcjonalna, niewpływająca na pozostałe czynności.

\begin{center}
	\begin{tabular}{|l|p{5em}|p{30em}|}
	\hline 
	 & Wykonawca & Czynność \\ 
	\hline
	1 & Użytkownik & Włączenie programu. \\
	\hline
	2 & Program & Po włączeniu automatyczne przeszukiwanie sieci LAN w celu znalezienia innych urządzeń z zainstalowanym i włączonym programem. \\
	\hline
	3 & Program & Wyświetlenie listy nazw dostępnych urządzeń w sieci LAN. \\
	\hline
	4 & Użytkownik & Wybranie urządzenia, a następnie wybranie pliku do przesłania. \\
	\hline
	\hline 
	1 & Użytkownik & Odświeżenie listy dostępnych urządzeń w sieci LAN za pomocą przycisku odświeżania. \\ 
	\hline 
	2 & Użytkownik & Wybranie urządzenia z listy nazw urządzeń dostępnych w sieci LAN. \\ 
	\hline 
	3* & Użytkownik & Wybranie katalogu w celu wyświetlenia jego zawartości. \\
	\hline
	4 & Użytkownik & Wybranie pliku do pobrania na własne urządzenie. \\
	\hline
	\end{tabular}
	
	\begin{tabular}{|l|p{5em}|p{30em}|}
	\hline 
	 & Wykonawca & Czynność \\
	\hline
	1 & Użytkownik & Użycie przycisku dodania urządzenia poza siecią LAN. \\
	\hline
	2 & Program & Pokazanie miejsca pozwalającego wpisać IP drugiego urządzenia.\\
	\hline
	3 & Użytkownik & Wpisanie poprawnego IP drugiego urządzenia z włączonym programem. \\
	\hline
	4 & Użytkownik & Użycie przycisku rozpoczęcia połączenia z drugim urządzeniem. \\ 
	\hline
	5 & Program & Program łączy się z urządzeniem o określonym IP i wyświetla zawartość udostępnionego folderu na nowym ekranie, gdy połączenie zakończyło się pomyślnie. \\
	\hline
	\end{tabular} 
	
	\subsection{Scenariusze alternatywne}
	
	\begin{tabular}{|l|p{5em}|p{30em}|}
	\hline 
	 & Wykonawca & Czynność \\
	\hline
	1 & Użytkownik & Włączenie programu. \\
	\hline
	2 & Program & Próba przeszukania sieci przy braku połączenia z siecią. \\
	\hline
	3 & Program & Wyświetlenie komunikatu o braku połączenia. \\
	\hline
	\hline
	1 & Użytkownik & Użycie przycisku dodania urządzenia poza siecią LAN. \\
	\hline
	2 & Program & Pokazanie miejsca pozwalającego wpisać IP drugiego urządzenia.\\
	\hline
	3 & Użytkownik & Wpisanie poprawnego IP drugiego urządzenia z włączonym programem. \\
	\hline
	4 & Użytkownik & Użycie przycisku rozpoczęcia połączenia z drugim urządzeniem. \\ 
	\hline
	5 & Program & Brak połączenia z Internetem jest sygnalizowane użytkownikowi komunikatem. \\
	\hline
	\end{tabular} 
\end{center}

\section{Wymagania pozafunkcjonalne}

\begin{enumerate}
	\item Program pozwala wyszukać inne urządzenia posiadające działający program.
	\item Program pozwala przeszukiwać foldery, które zostały udostępnione w programie na innych urządzeniach.
	\item Program pozwala zobaczyć udostępnione dane tylko z jednego urządzenia w danej chwili.
	\item Program pozwala się połączyć z innym urządzeniem, które nie znajduje się w tej samej sieci LAN, po wprowadzeniu IP drugiego urządzenia. Jednocześnie zakładając, że porty są otwarte.
	\item Program przesyła pliki za pomocą sieci LAN.
	\item Program pozwala pobierać udostępnione w programie pliki na innym urządzeniu na urządzenie użytkownika.
	

\end{enumerate}

\section{Makiety}

\end{document}